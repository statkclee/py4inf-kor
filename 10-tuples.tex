% LaTeX source for ``Python for Informatics: Exploring Information''
% Copyright (c)  2010-  Charles R. Severance, All Rights Reserved

\chapter{튜플(Tuples)}
\label{tuplechap}

\section{튜플은 불변이다.}

\index{튜플 (tuple)}
\index{자료형 (type)!튜플 (tuple)}
\index{순서 (sequence)}

튜플(tuple)\footnote{재미난 사실: 단어 ''튜플(tuple)''은 가변 길이 (한배, 두배, 세배, 네배, 다섯배, 여섯배, 일곱배 등) 숫자열에 
붙여진 이름에서 유래한다.}은 리스트와 마찬가지로 순서(sequence) 값이다. 
튜플에 저장된 값은 임의 자료형(type)이 될 수 있고, 정수로 색인 된다.
중요한 차이점은 튜플은 {\bf 불변(immutable)}하다는 것이다.
튜플은 또한 {\bf 비교 가능(comparable)}하고 {\bf 해쉬형(hashable)}이다.
따라서, 리스트 값을 정렬할 수 있고, 파이썬 딕셔너리 키 값으로 튜플을 사용할 수 있다.

\index{가변성 (mutability)}
\index{해쉬형의(hashable)}
\index{비교 가능 (comparable)}
\index{불변성 (immutability)}

구문론적으로, 튜플은 콤마로 구분되는 리스트 값이다.

\beforeverb
\begin{verbatim}
>>> t = 'a', 'b', 'c', 'd', 'e'
\end{verbatim}
\afterverb
%

꼭 필요하지는 않지만, 파이썬 코드를 봤을 때, 가독성을 높여 튜플을 빠르게 알아볼 수 있도록 괄호로 튜플을 감싸는 것이 일반적이다.

\index{괄호 (parentheses)!튜플 (tuples in)}

\beforeverb
\begin{verbatim}
>>> t = ('a', 'b', 'c', 'd', 'e')
\end{verbatim}
\afterverb
%

단일 요소를 가진 튜플을 생성하기 위해서 마지막 콤마를 포함해야 한다.

\index{싱글톤 (singleton)}
\index{튜플 (tuple)!싱글톤 (singleton)}

\beforeverb
\begin{verbatim}
>>> t1 = ('a',)
>>> type(t1)
<type 'tuple'>
\end{verbatim}
\afterverb
%

콤마가 없는 경우 파이썬에서는 \verb"('a')"을 괄호를 가진 문자열 표현으로 간주하여 문자열로 평가한다.

\beforeverb
\begin{verbatim}
>>> t2 = ('a')
>>> type(t2)
<type 'str'>
\end{verbatim}
\afterverb
%

튜플을 구축하는 다른 방법은 내장함수 {\tt tuple}을 사용하는 것이다. 
인자가 없는 경우, 빈 튜플을 생성한다.

\index{튜플 함수 (tuple function)}
\index{함수 (function)!튜플 (tuple)}

\beforeverb
\begin{verbatim}
>>> t = tuple()
>>> print t
()
\end{verbatim}
\afterverb
%

만약 인자가 문자열, 리스트 혹은 튜플 같은 순서(sequence)인 경우, 
{\tt tuple}에 호출한 결과는 요소 순서(sequence)를 가진 튜플이 된다.

\beforeverb
\begin{verbatim}
>>> t = tuple('lupins')
>>> print t
('l', 'u', 'p', 'i', 'n', 's')
\end{verbatim}
\afterverb
%

{\tt 튜플(tuple)}이 생성자 이름이기 때문에 변수명으로 튜플 사용을 피해야 한다.

대부분의 리스트 연산자는 튜플에서도 사용 가능하다. 
꺾쇠 연산자가 요소를 색인한다.

\index{꺾쇠 연산자 (bracket operator)}
\index{연산자 (operator)!꺾쇠 (bracket)}

\beforeverb
\begin{verbatim}
>>> t = ('a', 'b', 'c', 'd', 'e')
>>> print t[0]
'a'
\end{verbatim}
\afterverb
%

그리고, 슬라이스 연산자(slice operator)는 요소 범위를 선택한다.

\index{슬라이스 연산자 (slice operator)}
\index{연산자 (operator)!슬라이스 (slice)}
\index{튜플 (tuple)!슬라이스 (slice)}
\index{슬라이스 (slice)!튜플 (tuple)}

\beforeverb
\begin{verbatim}
>>> print t[1:3]
('b', 'c')
\end{verbatim}
\afterverb
%

하지만, 튜플 요소 중 하나를 변경하고 하면, 오류가 발생한다.

\index{예외 (exception)!자료형 오류 (TypeError)}
\index{자료형 오류 (TypeError)}
\index{항목 대입 (item assignment)}
\index{대입 (assignment)!항목 (item)}

\beforeverb
\begin{verbatim}
>>> t[0] = 'A'
TypeError: object doesn't support item assignment
\end{verbatim}
\afterverb
%

튜플 요소를 변경할 수는 없지만, 튜플을 다른 튜플로 교체는 할 수 있다.

\beforeverb
\begin{verbatim}
>>> t = ('A',) + t[1:]
>>> print t
('A', 'b', 'c', 'd', 'e')
\end{verbatim}
\afterverb
%

\section{튜플 비교하기}

\index{비교 (comparison)!튜플 (tuple)}
\index{튜플 (tuple)!비교 (comparison)}
\index{정렬 메쏘드 (sort method)}
\index{메쏘드 (method)!정렬 (sort)}

비교 연산자는 튜플과 다른 순서(sequence)와 함께 쓸 수 있다. 
파이썬은 각 순서(sequence)에서 비교를 첫 요소부터 시작한다.
만약 두 요소가 같다면, 다음 요소 비교를 진행하며 서로 다른 요소를 찾을 때까지 계속한다. 
후속 요소가 아무리 큰 값이라고 하더라도 비교 고려대상에서 제외된다.

\beforeverb
\begin{verbatim}
>>> (0, 1, 2) < (0, 3, 4)
True
>>> (0, 1, 2000000) < (0, 3, 4)
True
\end{verbatim}
\afterverb
%

{\tt 정렬 (sort)} 함수도 동일한 방식으로 동작한다.
첫 요소를 먼저 정렬하지만, 동일한 경우 두 번째 요소를 정렬하고, 그 후속 요소를 동일한 방식으로 정렬한다.

이 기능이 다음 {\bf DSU}라고 불리는 패턴이 된다.

\begin{description}

\item[데코레이트 (Decorate)] : 순서(sequence)에서 요소 앞에 하나 혹은 그 이상의 키를 가진 튜플 리스트를 구축해서 순서(sequence)를 장식한다.

\item[정렬 (Sort)] : 파이썬 내장 함수 {\tt sort}를 사용한 튜플 리스트를 정렬한다.

\item[언데코레이트 (Undecorate)] : 순서(sequence)의 정렬된 요소만 추출하여 장식을 지웁니다.

\end{description}

\label{DSU}
\index{DSU 패턴 (DSU pattern)}
\index{패턴 (pattern)!DSU}
\index{decorate-sort-undecorate pattern}
\index{패턴 (pattern)!decorate-sort-undecorate}
\index{로미오와 쥴리엣 (Romeo and Juliet)}

예를 들어, 단어 리스트가 있고 가장 긴 단어부터 가장 짧은 단어 순으로 정렬한다고 가정하자.

\beforeverb
\begin{verbatim}
txt = 'but soft what light in yonder window breaks'
words = txt.split()
t = list()
for word in words:
   t.append((len(word), word))

t.sort(reverse=True)

res = list()
for length, word in t:
    res.append(word)

print res
\end{verbatim}
\afterverb
%

첫번째 루프는 튜플 리스트를 생성하고, 각 튜플은 단어 앞에 길이 정보를 가진다.

{\tt 정렬(sort)}함수는 첫번째 요소, 길이를 우선 비교하고, 동률일 경우에만 두 번째 요소를 고려한다.
{\tt 정렬(sort)} 함수의 인자 {\tt reverse=True}는 내림차순으로 정렬한다는 의미다.

\index{예약어 인자 (keyword argument)}
\index{인자 (argument)!예약어 (keyword)}
\index{운행법 (traversal)}

두 번째 루프는 튜플 리스트를 운행하여 훑고, 길이에 따라 내림차순으로 단어 리스트를 생성한다.
그래서, 5 문자 단어는 역 알파벳 순으로 정렬되어 있다. 
다음 리스트에서 ``what''이 ``soft'' 보다 앞에 나타난다.

프로그램의 출력은 다음과 같다.

%
\beforeverb
\begin{verbatim}
['yonder', 'window', 'breaks', 'light', 'what', 
'soft', 'but', 'in']
\end{verbatim}
\afterverb
%

물론, 파이썬 리스트로 변환하여 내림차순으로 정렬된 문장은 시적인 의미를 많이 잃어버렸다.

\section{ 튜플 대입(Tuple Assignment)}
\label{tuple assignment}

\index{튜플 (tuple)!대입 (assignment)}
\index{대입 (assignment)!튜플 (tuple)}
\index{교환 패턴 (swap pattern)}
\index{패턴 (pattern)!교환 (swap)}

파이썬 언어의 독특한 구문론적인 기능중의 하나는 대입문의 왼편에 튜플을 놓을 수 있다는 것이다. 
왼편이 순서(sequence)인 경우 한번에 하나 이상의 변수에 대입할 수 있다.

다음 예제에서, 순서(sequence)인 두개 요소를 갖는 리스트가 있다. 
하나의 문장으로 순서(sequence)의 첫번째와 두번째 요소를 변수 {\tt x}와 {\tt y}에 대입한다. 

\beforeverb
\begin{verbatim}
>>> m = [ 'have', 'fun' ]
>>> x, y = m
>>> x
'have'
>>> y
'fun'
>>> 
\end{verbatim}
\afterverb
%

마술이 아니다. 파이썬은 \emph{대략} 튜플 대입 구문을 다음과 같이 해석한다.\footnote{파이썬은 구문을 문자 그대로 해석하지는 않는다.
예를 들어, 동일한 것을 딕셔너리로 작성한다면, 기대한 것처럼 동작하지는 않는다.}

\beforeverb
\begin{verbatim}
>>> m = [ 'have', 'fun' ]
>>> x = m[0]
>>> y = m[1]
>>> x
'have'
>>> y
'fun'
>>> 
\end{verbatim}
\afterverb

스타일적으로 대입문 왼편에 튜플을 사용할 때, 괄호를 생략한다. 
하지만 다음은 동일하게 적합한 구문이다.

\beforeverb
\begin{verbatim}
>>> m = [ 'have', 'fun' ]
>>> (x, y) = m
>>> x
'have'
>>> y
'fun'
>>> 
\end{verbatim}
\afterverb
%

튜플 대입문을 사용하는 특히 똑똑한 응용사례는 단일 문장으로 두 변수 값을 {\bf 교환(swap)}하는 것이다.

\beforeverb
\begin{verbatim}
>>> a, b = b, a
\end{verbatim}
\afterverb
%

양쪽 문장이 모두 튜플이지만, 왼편은 튜플 변수이고 오른편은 튜플 표현식이다.
오른편 각각의 값이 왼편 해당 변수에 대입된다. 
대입이 이루어지기 전에 오른편의 모든 표현식이 평가된다.

왼편의 변수 갯수와 오른편의 값의 갯수는 동일해야 한다.

\index{예외 (exception)!값 오류 (ValueError)}
\index{값 오류 (ValueError)}

\beforeverb
\begin{verbatim}
>>> a, b = 1, 2, 3
ValueError: too many values to unpack
\end{verbatim}
\afterverb
%

좀더 일반적으로, 오른편은 임의 순서(문자열, 리스트 혹은 튜플)가 될 수 있다.
예를 들어, 전자우편 주소를 사용자 이름과 도메인으로 분할하기 위해서 다음과 같이 프로그램을 작성할 수 있다.

\index{분할 메쏘드 (split method)}
\index{메쏘드 (method)!분할 (split)}
\index{전자우편 주소 (email address)}

\beforeverb
\begin{verbatim}
>>> addr = 'monty@python.org'
>>> uname, domain = addr.split('@')
\end{verbatim}
\afterverb
%

{\tt 분할 (split)} 함수로부터 반환되는 값은 두개 요소를 가진 리스트다. 
첫번째 요소는 {\tt uname}에 두번째 요소는 {\tt domain}에 대입된다.

\beforeverb
\begin{verbatim}
>>> print uname
monty
>>> print domain
python.org
\end{verbatim}
\afterverb
%

\section{딕셔너리와 튜플}

\index{딕셔너리 (dictionary)}
\index{항목 메쏘드 (items method)}
\index{메쏘드 (method)!항목 (items)}
\index{키-값 페어 (key-value pair)}

딕셔너리에는 튜플 리스트를 반환하는 {\tt items} 메쏘드가 있다. 
각 튜플은 키-값 페어(key-value pair)다.\footnote{파이썬 3.0으로 가면서 살짝 달라졌다.}

\beforeverb
\begin{verbatim}
>>> d = {'a':10, 'b':1, 'c':22}
>>> t = d.items()
>>> print t
[('a', 10), ('c', 22), ('b', 1)]
\end{verbatim}
\afterverb
%

딕셔너리로부터 기대했듯이, 항목은 특별한 순서가 없다.

하지만 튜플 리스트는 리스트여서 비교가 가능하기 때문에, 튜플 리스트를 정렬할 수 있다.
딕셔너리를 튜플 리스트로 변환하는 것이 키로 정렬된 딕셔너리 내용을 출력할 수 있게 한다.

\beforeverb
\begin{verbatim}
>>> d = {'a':10, 'b':1, 'c':22}
>>> t = d.items()
>>> t
[('a', 10), ('c', 22), ('b', 1)]
>>> t.sort()
>>> t
[('a', 10), ('b', 1), ('c', 22)]
\end{verbatim}
\afterverb
%

새로운 리스트는 키 값으로 오름차순 알파벳 순으로 정렬된다.

\section{딕셔너리로 다중 대입}

\index{운행 (traverse)!사전 (dictionary)}
\index{사전 (dictionary)!운행 (traversal)}

{\tt items} 함수, 튜플 대입, {\tt for}문을 조합해서, 
단일 루프로 딕셔너리의 키와 값을 운행하여 훑는 멋진 코드 패턴을 만들 수 있다.

\beforeverb
\begin{verbatim}
for key, val in d.items():
    print val, key
\end{verbatim}
\afterverb
%

상기 루프에는 두개의 {\bf 반복 변수(iteration variables)}가 있다.
{\tt items} 함수가 튜플 리스트를 반환하고, {\tt key, val}는 튜플 대입하여 딕셔너리에 있는 각각의 키-값 페어(key-value pair)를 성공적으로 반복한다.

매번 루프를 반복할 때마다, {\tt key}와 {\tt value}는 딕셔너리(여전히 해쉬 순으로 되어 있음)의 다음 키-값 페어(key-value pair)로 진행한다.

루프의 출력결과는 다음과 같다.

\beforeverb
\begin{verbatim}
10 a
22 c
1 b
\end{verbatim}
\afterverb
%

다시 한번 해쉬 키 순서다. (즉, 특별한 순서가 없다.)

두 기술을 조합하면, 딕셔너리 내용을 키-값 페어(key-value pair)에 저장된 \emph{값}의 순서로 정렬하여 출력할 수 있다.

이것을 수행하기 위해서, 각 튜플이 {\tt (value, key)} 형태인 튜플 리스트를 작성한다. 
{\tt items} 메쏘드를 사용하여 리스트 {\tt (key, value)} 튜플을 만든다. 
하지만 이번에는 키가 아닌 값으로 정렬한다.
키-값(key-value) 튜플 리스트를 생성하면, 역순으로 리스트를 정렬하고 새로운 정렬 리스트를 출력하는 것은 쉽다.

\beforeverb
\begin{verbatim}
>>> d = {'a':10, 'b':1, 'c':22}
>>> l = list()
>>> for key, val in d.items() :
...     l.append( (val, key) )
... 
>>> l
[(10, 'a'), (22, 'c'), (1, 'b')]
>>> l.sort(reverse=True)
>>> l
[(22, 'c'), (10, 'a'), (1, 'b')]
>>> 
\end{verbatim}
\afterverb
%

조심스럽게 각 튜플 첫번째 요소로 값(value)을 갖는 튜플 리스트를 생성했다. 
튜플 리스트를 정렬하여 값으로 정렬된 딕셔너리가 생성되었다.

\section{가장 빈도수가 높은 단어}

\index{로미오와 쥴리엣 (Romeo and Juliet)}

\emph{로미오와 쥴리엣 2장 2막} 텍스트 파일로 다시 돌아와서, 텍스트에서 가장 빈도수가 높은 단어를 10개를 출력하기 위해서
상기 학습한 기법을 사용하여 프로그램을 보강해보자.

\beforeverb
\begin{verbatim}
import string
fhand = open('romeo-full.txt')
counts = dict()
for line in fhand:
    line = line.translate(None, string.punctuation)
    line = line.lower()
    words = line.split()
    for word in words:
        if word not in counts:
            counts[word] = 1
        else:
            counts[word] += 1

# Sort the dictionary by value
lst = list()
for key, val in counts.items():
    lst.append( (val, key) )

lst.sort(reverse=True)

for key, val in lst[:10] :
    print key, val
\end{verbatim}
\afterverb
%

파일을 읽고 각 단어를 문서의 단어 빈도수에 매핑(사상)하여 딕셔너리를 계산하는 프로그램 첫 부분은 바뀌지 않는다.
하지만, {\tt counts} 를 단순히 출력하는 대신에 {\tt (val, key)} 튜플 리스트를 생성하고 역순으로 리스트를 정렬한다.

값이 첫 위치에 있기 때문에, 비교 목적으로 값을 사용한다. 
만약 동일한 값을 가진 튜플이 하나이상 존재한다면, 두번째 요소(키, key)를 살펴본다.
그래서 값이 동일한 경우 키의 알파벳 순으로 추가 정렬된다.

마지막에 다중 대입 반복을 수행하는 멋진 {\tt for} 루프를 작성한다. 
그리고, 리스트 슬라이스({\tt lst[:10]})를 통해 가장 빈도수가 높은 상위 10개 단어를 출력한다.

이제 단어 빈도 분석을 위해서 작성한 프로그램의 마지막 출력결과는 원하는 바를 완수한 것처럼 보인다.

\beforeverb
\begin{verbatim}
61 i
42 and
40 romeo
34 to
34 the
32 thou
32 juliet
30 that
29 my
24 thee
\end{verbatim}
\afterverb
%

복잡한 데이터 파싱과 분석 작업이 이해하기 쉬운 19줄 파이썬 프로그램으로 수행된 사실이 왜 파이썬이 정보 탐색 언어로서 좋은 선택인지 보여준다.

\section{ 딕셔너리 키로 튜플 사용하기}

\index{튜플 (tuple)!딕셔너리 키 (as key in dictionary)}
\index{해쉬형(hashable)}

튜플은 {\bf 해쉬형(hashable)}이고, 리스트는 그렇지 못하기 때문에, 만약 딕셔너리에 사용할 {\bf 복합(composite)}키를 생성하려면, 키로 튜플을 사용해야 한다.

만약 성(last-name)과 이름(first-name)을 가지고 전화번호에 사상(매핑, mapping)하는 전화번호부를 생성하려고 하면, 복합키와 마주친다.
변수 {\tt last}, {\tt first}, {\tt number}을 정의했다고 가정하면, 다음과 같이 딕셔너리 대입문을 작성할 수 있다.

\beforeverb
\begin{verbatim}
directory[last,first] = number
\end{verbatim}
\afterverb
%

꺾쇠 괄호 표현은 튜플이다. 
딕셔너리를 훑기 위해서 {\tt for} 루프에 튜플 대입을 사용한다.

\index{튜플 (tuple)!꺾쇠 (in brackets)}

\beforeverb
\begin{verbatim}
for last, first in directory:
    print first, last, directory[last,first]
\end{verbatim}
\afterverb
%

상기 루프가 튜플인 {\tt directory}에 키를 훑는다. 
각 튜플 요소를 {\tt last}, {\tt first}에 대입하고 나서, 이름과 해당 전화번호를 출력한다.

\section{순서(sequence) : 문자열, 리스트, 튜플}
\index{순서 (sequence)}

여기서 리스트 튜플에 초점을 맞추었지만, 이장의 거의 모든 예제가 또한 리스트의 리스트, 튜플의 튜플, 리스트 튜플에도 동작한다.
가능한 모든 조합을 열거하는 것을 피하기 위해서, 순서의 순서(sequences of sequences)에 대해서 논의하는 것이 때로는 쉽다.

대부분의 문맥에서 다른 종류의 순서(문자열, 리스트, 튜플)는 상호 호환해서 사용될 수 있다. 
그런데 왜 그리고 어떻게 다른 것보다 이것을 선택해야 될까?

\index{문자열 (string)}
\index{리스트 (list)}
\index{튜플 (tuple)}
\index{변경성 (mutability)}
\index{불변성 (immutability)}

명확한 것부터 시작하자.
문자열은 요소가 문자여야 하기 때문에 다른 순서(sequence)보다 제약이 따른다.
또한 문자열은 불변(immutable)이다. 
새로운 문자열을 생성하는 것과 반대로, 문자열에 있는 문자를 변경하고자 한다면, 
대신에 문자 리스트를 사용하는 것을 생각할 수 있다.

리스트는 튜플보다 좀더 일반적으로 사용된다. 
이유는 대체로 변경가능(mutable)하기 때문이다.
하지만, 다음 몇가지 경우에 튜플이 좀더 선호된다.

\begin{enumerate}

\item {\tt return} 문처럼 어떤 맥락에서, 리스트보다 튜플을 생성하는 것이 구문론적으로 더 간략하다.
다른 맥락에서는 리스트가 더 선호될 수 있다.

\item 딕셔너리 키로서 순서(sequence)를 사용하려면, 튜플이나 문자열같은 불변 자료형(immutable type)을 사용해야 한다.

\item 함수에 인자로 순서(sequence)를 전달하려면, 튜플을 사용하는 것이 에일리어싱(aliasing)으로 생기는 예기치 못한 행동에 대한 가능성을 줄여 준다.

\end{enumerate}

튜플은 불변(immutable)이어서, 기존 리스트를 변경하는 {\tt sort}, {\tt reverse} 같은 메쏘드를 제공하지는 않는다.
하지만, 파이썬이 제공하는 내장함수 {\tt sorted}, {\tt reversed} 를 통해서, 매개 변수로 임의 순서(sequence)를 전달 받아서, 같은 요소를 다른 순서로 정렬된 새로운 리스트를 반환한다.

\index{정렬 함수 (sorted function)}
\index{함수 (function)!정렬 (sorted)}
\index{역 함수 (reversed function)}
\index{함수 (function)!역 (reversed)}

\section{디버깅}

\index{디버깅 (debugging)}
\index{자료 구조 (data structure)}
\index{모양 오류 (shape error)}
\index{오류 (error)!모양 (shape)}

리스트, 딕셔너리, 튜플은 {\bf 자료 구조(data structures)}로 일반적으로 알려져 있다.
이번장에서 리스트 튜플, 키로 튜플, 값으로 리스트를 담고 있는 딕셔너리 같은 복합 자료 구조를 보기 시작했다.
복합 자료 구조는 유용하지만, 저자가 작명한 {\bf 모양 오류(shape errors)}라고 불리는 오류에 노출되어 있다.
즉, 자료 구조가 잘못된 자료형(type), 크기, 구성일 경우 오류가 발생한다. 
혹은 코드를 작성하고, 자료의 모양이 생각나지 않는 경우도 오류의 원인이 된다.

예를 들어, 정수 하나를 가진 리스트를 기대하고, (리스트가 아닌) 일반 정수를 넘긴다면, 작동하지 않는다.

프로그램을 디버깅할 때, 정말 어려운 버그를 잡으려고 작업을 한다면, 다음 네가지를 시도할 수 있다.

\begin{description}

\item[코드 읽기(reading):] 
코드를 면밀히 조사하고, 스스로에게 다시 읽어 주고, 코드가 자신이 작성한 의도를 담고 있는지  점검하라.

\item[실행(running):] 
변경해서 다른 버젼을 실행해서 실험하라. 
종종, 프로그램이 적절한 곳에 적절한 것을 보여준다면, 문제가 명확하다. 
발판(scaffolding)을 만들기 위해서 때때로 시간을 들일 필요도 있다.

\item[반추(ruminating):] 
생각의 시간을 갖자. 
어떤 종류의 오류인가: 구문, 실행, 의미론(semantic). 
오류 메시지로부터 혹은 프로그램 출력결과로부터 무슨 정보를 얻을 수 있는가?
어떤 종류 오류가 지금 보고 있는 문제를 만들었을까? 
문제가 나타나기 전에, 마지막으로 변경한 것은 무엇인가?

\item[퇴각(retreating):]
어느 시점에선가, 최선은 물러서서, 최근의 변경을 다시 원복하는 것이다. 
잘 동작하고 이해하는 프로그램으로 다시 돌아가서, 다시 프로그램을 작성한다.
\end{description}

초보 프로그래머는 종종 이들 활동 중 하나에 사로잡혀 다른 것을 잊곤 한다. 
활동 각각은 고유한 실패 방식과 함께 온다.

\index{인쇄 오류 (typographical error)}

예를 들어, 프로그램을 정독하는 것은 문제가 인쇄상의 오류에 있다면 도움이 되지만, 문제가 개념상 오해에 뿌리를 두고 있다면 그다지 도움이 되지 못한다. 
만약 작성한 프로그램을 이해하지 못한다면, 100번 읽을 수는 있지만, 오류를 발견할 수는 없다.
왜냐하면, 오류는 여러분 머리에 있기 때문입니다.

\index{실험 디버깅 (experimental debugging)}

만약 작고 간단한 테스트를 진행한다면, 실험을 수행하는 것이 도움이 될 수 있다.
하지만, 코드를 읽지 않거나, 생각없이 실험을 수행한다면, 프로그램이 작동될 때까지 무작위 변경하여 개발하는 ''랜덤 워크 프로그램(random walk programming)'' 패턴에 빠질 수 있다. 
말할 필요없이 랜덤 워크 프로그래밍은 시간이 오래 걸린다.

\index{랜덤 워크 프로그래밍 (random walk programming)}
\index{개발 계획 (development plan)!랜덤 워크 프로그래밍 (random walk programming)}

생각할 시간을 가져야 한다. 
디버깅은 실험 과학 같은 것이다. 
문제가 무엇인지에 대한 최소한 한 가지 가설을 가져야 한다.
만약 두개 혹은 그 이상의 가능성이 있다면, 이러한 가능성 중에서 하나라도 줄일 수 있는 테스트를 생각해야 한다.

휴식 시간을 가지는 것은 생각하는데 도움이 된다. 
대화를 하는 것도 도움이 된다.
문제를 다른 사람 혹은 자신에게도 설명할 수 있다면, 질문을 마치기도 전에 답을 종종 발견할 수 있다.

하지만, 오류가 너무 많고 수정하려는 코드가 매우 크고, 복잡하다면 최고의 디버깅 기술도 무용지물이다.
가끔, 최선의 선택은 퇴각하는 것이다. 
작동하고 이해하는 곳까지 후퇴해서 프로그램을 간략화하라.

초보 프로그래머는 종종 퇴각하기를 꺼려한다. 
왜냐하면, 설사 잘못되었지만, 한줄 코드를 지울 수 없기 때문이다.
삭제하지 않는 것이 기분을 좋게 한다면, 다시 작성하기 전에 프로그램을 다른 파일에 복사하라.
그리고 나서, 한번에 조금씩 붙여넣어라. 

정말 어려운 버그(hard bug)를 발견하고 고치는 것은 코드 읽기, 실행, 반추, 때때로 퇴각을 요구한다.
만약 이들 활동 중 하나도 먹히지 않는다면, 다른 것들을 시도해 보세요.

\section{용어정의}

\begin{description}

\item[비교가능한(comparable):] 동일한 자료형의 다른 값과 비교하여 큰지, 작은지, 혹은 같은지를 확인하기 위해서 확인할 수 있는 자료형(type).
비교가능한(comparable) 자료형은 리스트에 넣어서 정렬할 수 있다.
\index{비교가능한 (comparable)}

\item[자료 구조(data structure):] 연관된 값의 집합, 종종 리스트, 딕셔너리, 튜플 등으로 조직화된다.
\index{자료 구조 (data structure)}

\item[DSU:] ``decorate-sort-undecorate,''의 약어로 리스트 튜플을 생성, 정렬, 결과 일부 추출을 포함하는 패턴.
\index{DSU 패턴 (DSU pattern)}

\item[모음(gather):] 가변-길이 인자 튜플을 조합하는 연산.
\index{모음 (gather)}

\item[해쉬형(hashable):] 해쉬 함수를 가진 자료형(type). 
정수, 소수점, 문자열 같은 불변형은 해쉬형이다. 
리스트나 딕셔너리 처럼 변경가능한 형은 해쉬형이 아니다.
\index{해쉬형 (hashable)}

\item[스캐터(scatter):] 순서(sequence)를 리스트 인자로 다루는 연산.
\index{스캐터(scatter)}

\item[(자료 구조의) 모양 (shape (of a data structure)):] 자료 구조의 자료형(type), 크기, 구성을 요약.
\index{모양 (shape)}

\item[싱글톤(singleton):] 단일 요소를 가진 리스트 (혹은 다른 순서(sequence)).
\index{싱글톤 (singleton)}

\item[튜플(tuple):] 불변 요소들의 순서 (sequence).
\index{튜플 (tuple)}

\item[튜플 대입(tuple assignment):] 오른편 순서(sequence)와 왼편 튜플 변수를 대입.
오른편이 평가되고나서 각 요소들은 왼편의 변수에 대입된다.
\index{튜플 대입 (tuple assignment)}
\index{대입 (assignment)!튜플 (tuple)}

\end{description}


\section{연습문제}

\begin{ex}
앞서 작성한 프로그램을 다음과 같이 수정하세요.
''From''라인을 읽고 파싱하여 라인에서 주소를 뽑아내세요.
딕셔너리를 사용하여 각 사람으로부터 메시지 숫자를 계수(count)한다.

모든 데이터를 읽은 후에 가장 많은 커밋(commit)을 한 사람을 출력하세요.
딕셔너리로부터 리스트 (count, email) 튜플을 생성하고 역순으로 리스트를 정렬한 후에 가장 많은 커밋을 한 사람을 출력하세요.

\beforeverb
\begin{verbatim}
Sample Line:
From stephen.marquard@uct.ac.za Sat Jan  5 09:14:16 2008

Enter a file name: mbox-short.txt
cwen@iupui.edu 5

Enter a file name: mbox.txt
zqian@umich.edu 195
\end{verbatim}
\afterverb
\end{ex}
\begin{ex}

이번 프로그램은 각 메시지에 대한 하루 중 시간의 분포를 계수(count)한다.
''From'' 라인으로부터 시간 문자열을 찾고 콜론(:) 문자를 사용하여 문자열을 쪼개서 시간을 추출합니다.
각 시간별로 계수(count)를 누적하고 아래에 보여지듯이 시간 단위로 정렬하여 한 라인에 한시간씩 계수(count)를 출력합니다.

\beforeverb
\begin{verbatim}
Sample Execution:
python timeofday.py
Enter a file name: mbox-short.txt
04 3
06 1
07 1
09 2
10 3
11 6
14 1
15 2
16 4
17 2
18 1
19 1
\end{verbatim}
\afterverb
\end{ex}


\begin{ex}
파일을 읽고, 빈도(frequencey)에 따라 내림차순으로 {\em 문자(letters)}를 출력하는 프로그램을 작성하세요.
작성한 프로그램은 모든 입력을 소문자로 변환하고 a-z 문자만 계수(count)한다. 
공백, 숫자, 문장기호 a-z를 제외한 다른 어떤 것도 계수하지 않습니다.
다른 언어로 구성된 텍스트 샘플을 구해서 언어마다 문자 빈도가 어떻게 변하는지 살펴보세요.
결과를 \url{wikipedia.org/wiki/Letter_frequencies} 표와 비교하세요.

\index{문자 빈도 (letter frequency)}
\index{빈도 (frequency)!문자 (letter)}

\end{ex}

